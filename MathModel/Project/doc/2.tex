
\section{Описание модели}

 После придания грузу импульса $S$ система покидает положение равновесия. Груз отклоняется от начального положения на отрицательный угол $x(t)$, относительно шарнира, с отрицательной угловой скоростью $\dot{x}(t)$. В то же время на балку начинает действовать сила упругости пружины, стремящаяся вернуть систему в положения равновесия, то есть груз движется с положительным угловым ускорением $\ddot{x}(t)$. В данной модели так же можно рассматривать действие силы тяжести груза, но для упрощения вычислений сила тяжести в данной модели не учитывается. Как только $\dot{x}(t)=0$ груз начнет движение к положению равновесия. Когда $x(t)=0$ и следовательно $\ddot{x}(t)=0$, по инерции груз продолжит движение вверх по дуге с положительной угловой скоростью $\dot{x}(t)$. А сила упругости пружины придаст грузу отрицательное угловое ускорение $\ddot{x}(t)$. В итоге видно, что система подобна пружинному маятнику. 
 
 Для данной модели для упрощения вычислений $\Delta \vec{d}(t)$ -- величина отклонения пружины от положения равновесия измеряется длиной дуги, полученной перемещением точки закрепления пружины к балке по ходу ее движения, так же $\vec{F}_{\text{упр.}}(t)$ пружины считается направленной параллельно $\vec{a}(t)$ - ускорению груза.  
 
 Для описания движения балки воспользуемся вторым законом Ньютона и правилом рычагов:
  $$ \vec{F}_{\text{упр.}}(t) = 2\vec{F}_{\text{равн.}}(t), $$
где $\vec{F}_{\text{равн.}}(t)$ - равнодействующая сила груза. Далее имеем, 
  $$ \Delta \vec{d}(t) c_0 = 2 \vec{a}(t) m. $$
И по определениям $\Delta \vec{d}(t)$ и углового ускорения,
  $$ - x(t) L c_0 = 2 \cdot \ddot{x}(t) \cdot 2 L m  \text{\quad или \quad} 4 m \ddot{x}(t) + c_0 x(t) = 0 $$ 

 К полученному дифференциальному уравнению несложно найти начальные условия для формирования задачи Коши, а именно,
  $$ x(0) = 0 \text{\quad и \quad} \dot{x}(0) = \frac{S}{2mL}. $$ 
Решив поставленную задачу, выведем прямую зависимость отклонения системы от времени, что и необходимо при визуализации модели.







